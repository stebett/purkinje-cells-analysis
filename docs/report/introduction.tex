\section{Introduction}

This work is a continuation of a previous investigation on spiking data from rats. The recordings has been done with multiple tetrodes implanted in the cerebellum of the animals, while they were freely moving. The analyses are focused on the characteristics of the spike trains at the moment when the rats were reaching and grasping an object. 

The previous analyses mostly investigated the differences between spiketrains emitted by neighbor neurons and distant neurons. Cerebellum is known for being topographycally and functionally organized, hence we expect to see that neurons that are close to each other show similar activity. This was in fact confirmed, as it will be shown, but the question if correlated spike trains convey more information than spike trains considered as indipendent stays open.

Another interesting result from the previous analyses is that neighbor neurons show synchrony on a millisecond time scale, which suggests a possible dynamic to further investigate.

