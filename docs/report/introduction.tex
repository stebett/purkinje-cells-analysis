\section{Introduction}

It is well known that the cerebellum plays an important role in coordination of movements, thanks to its interactions with the motor cortex. Purkinje cells, the sole output neurons in the cerebellar cortex, are the subject of this study. 

The problem that cerebellum solves is that control of movement based purely on sensory input is hard to achieve, since the feedback is too slow.

Current theories suggest that an internal inverse model \cite{wolpert1998internal} could manage to achieve coordination. The idea is that the cerebellum, having knowledge of the state of the system, could be able to predict the right motor command so that the next state is the desired one.

This research focuses on the study of dynamics displayed by multi-cell activity. Neural population dynamics during reaching has been already investigated for motor cortex \cite{churchland2012neural}, but there is currently scarce knowledge on cerebellum from this perspective.
Understanding the computations carried out inside the cerebellum would be extremely valuable, especially for robotic motor control \cite{casellato2014adaptive}, and potentially also for BMI such as prosthetic limbs. 
The objective of this research is to show how Purkinje cells are modulated during the movement: in particular how Purkinje cell discharge encodes variations in the speed of movements, and how the co-variation of multiple cells relates to movement features.
In this preliminary work, an exploratory analysis has been performed, investigating the possibility of encoding of movement features in the low-dimensional trajectories of single-trial recordings.

The work presented here is based on previous unpublished research, where analyses were done in the direction of showing synchronicity at very short time scales between neighbor neurons. 
