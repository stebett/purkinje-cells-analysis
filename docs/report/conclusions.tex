\section{Conclusions and future work}

Rotational dynamics are consistently found in neuronal population activity during reaching \cite{churchland2012neural}. It may be possible that cerebellum dynamics has a causal relationship with the dynamics found in the motor cortex, but specific experimental paradigms have to be designed to investigate that.

Neural trajectories are hard to interpret, but can be useful as hypothesis generation tools.
In fact, a possible way to proceed in this work could be to come up with likely dynamical structures (e.g. attractors), then implement them through Recurrent Neural Networks (RNN), to then validate the hypothesis through simulations and comparison with biological data.

However, this process is arguably more an art than a technique, and requires a deep understanding of dynamical systems. A possible alternative would be to use Latent Factor Analysis via Dynamical Systems (LFADS) \cite{sussillo2016lfads}.
LFADS uses spike train data to infer the latent dynamics through fitting of RNNs, allowing to avoid the manual step of designing the artificial network.
