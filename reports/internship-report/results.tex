\section{Results}

\myparagraph{Paradigm}
The experimental paradigm consisted in a reach-to-grasp task carried out by 13 rats. The rats were freely moving and were trained to grasp a pellet of food with the forepaw and bring it to the mouth in order to eat it. The execution of the task was recorded using a 50Hz video camera, allowing the labeling of three landmarks for successive analyses: \emph{lift} of the forepaw from the ground, \emph{cover}, that is the time of contact of the forepaw with the pellet, and \emph{grasp}, when the animal closes the digits around the food. 
The average lift to cover, and cover to grasp intervals were 262±55 ms and 200±54 ms respectively (mean±sd). While the rats were executing the task, electrophysiological signal was recorded through one or more tetrodes. The recording sites were located on the cortex of the paramedian lobule in the cerebellum, that is known to be involved in movement of limbs, and the sites were chosen to be ipsi-lateral to the preferred limb used by the individual rats, since recruitment  of cells if ipsi-lateral. The tetrodes were implanted in the Purkinje layer, and they were distanced between each other at least 300$\mu$m, to be certain that neurons on different tetrodes belonged to different microzones. Almost every tetrode successfully recorded signals from multiple Purkinje cells, that were selected by rate (>25Hz) and presence of complex spikes. 
All the data production and spike-sorting was performed by Hong-Ying Gao before the beginning of my internship.

\myparagraph{Rate modulation during the task}
After cleaning the data (see Materials and Methods), we selected the neurons modulating their activity during the movement by averaging the firing rates around the landmarks between trials, and comparing them to the baseline. 121 out of 151 ($\sim$80\%) Purkinje cells were significantly modulated around at least one of the landmarks of the movement, either lift (n=93), cover (n=99), or grasp (n=92). Figure \ref{figure1}A shows the average normalized activity around lift: 93 cells were modulated, among which 48 significantly increased, 32 significantly decreased and 13 both increased and decreased their firing rate (Figure \ref{figure1}C). The majority of cells where modulated by every landmark (n=68), and a smaller share responded to two (n=27) or one(n=26) event, as can be observed from the multi-psth (Figure \ref{figure1}B). Modulation was considered significant when the mean spiking rate in an interval of 50ms was exceeded the baseline by 2.5 times.

\bigplot[analyses/figure1.png]{figure1}{
	\textbf{Purkinje cells in the paramedian lobule are modulated during a skilled reaching task}\\
	\textbf{A}: Peri-stimulus time histograms of all Purkinje cells around ``cover''. Color encodes the normalized change in firing rate: the spikes has been binned with binsize=1ms, convolved with a 10ms Gaussian kernel to calculate a continuous estimate of the instantaneous firing rate  and divided by the mean firing rate of their baseline activity, calculated between 5 and 3 seconds before the initiation of the movement, and finally averaged across trials. 
	\textbf{B}: The Multi-PSTH has been calculated aligning every spike trains on all 3 landmarks, by defining a fixed number of bins for each interval and stretching or compressing the binsize so that every spike train would fit. After that, the binned spike trains had been treated in the same way as in the PSTH. This process allows the visualisation of the evolution of activity between every landmark (l: lift, c: cover, g:grasp).
	\textbf{C}: Average time course of the normalized firing rate during ``cover'' for 3 populations of Purkinje cells: all Purkinje cells (in black), Purkinje cells which significantly increased (red) and decreased (blue) their firing rates during the task. 
\textbf{D}: Image of a rat while grasping a pellet of food. The forelimb of the rat needs to pass between two obstacles, so that the trajectory of the movement does not change between trials. Picture from McKenna and Whishaw, 1999.}

\myparagraph{Similarities of time course of firing}
To assess the significance of covariation of pairs of cells, we computed the correlation coefficient of their profile of firing rate (\emph{i.e.} their peri-stimulus histogram) around the landmark events (Figure \ref{figure2}B). On average, the time course of firing was positively correlated around the landmarks for the pairs of neighboring cells but not for pairs of distant cells recorded from different tetrodes (correlation coefficient for firing profile $\pm$150ms around \emph{cover}, mean $\pm$ sem). 
The average distance between recording sites for distant cells was $0.88 \pm 0.45$ mm (mean $\pm$ sd) with no preference for the parasagittal or coronal directions. However, assessing the variation of firing rate around an event over multiple trials with event-triggered averages is limited by the uncertainty of the event (the frame rate of the video used to identify the events was 50Hz, yielding a 20ms uncertainty on the actual timing of the event), and by the inter-trial variability in the speed of the execution of the movement around the landmark event, which reduces the relevance of the firing rate profile for long delays from the event used to align the trials. 

Overall, these results indicate that neighboring cells in the paramedian lobule exhibit weak but significantly similar firing rate profiles along the reach-and-grasp movement.

\bigplot[analyses/figure2.png]{figure2}{
	\textbf{Neighboring Purkinje cells are weakly co-modulated during task.}\\
	\textbf{A}: Examples of firing rate time course of simultaneously recorded pairs of neighboring cells (shown in red or blue lines), with high (top), intermediate (middle), and low (bottom) similarities of firing modulation, respectively. The spike density around the ``cover'' landmark was convoluted with a 10ms Gaussian kernel and normalized to the average over the 1 second window. \\
	\textbf{B}: Correlation coefficients of the firing rate profiles of pairs of neighboring (recorded from the same tetrode) or distant (recorded from different tetrodes) cells around the landmarks.
}

\myparagraph{Short-term correlations between pairs of Purkinje cells}
To assess the presence of synchrony between the responses of Purkinje cells we measured the cross-covariance of the spike trains, obtaining the cross-correlograms between spike trains. The cross-correlogram allows the quantification of the number of spikes of a target cell triggered at some temporal distance from the spikes of a reference cell, effectively measuring the time scale of synchrony between two spike trains. We observed a clear peak in the center of the correlogram, (Figure \ref{figure3}B) showing an increase of around 20\% of the spikes within the ±5ms interval when compared to the ±20ms window. This effect was amplified during modulation of the firing rate and was found only between couples of neurons belonging to the same microzone (Figure \ref{figure3}C). The fact that distant neurons do not show this peak even when cross-correlating the spike trains modulated chunks only, suggests that the increase in firing rate alone does not justify the peak in the cross-correlogram. 
Interestingly, couples of neurons that have similar firing rate profiles show the same cross-correlogram as couples of neurons that have different firing rate profiles (Figures \ref{figure3}F and G). This effect is apparently counterintuitive, because correlated spike trains should display synchrony, while uncorrelated spike trains should not. However, Purkinje cells need to perform two functions at the same time: to send a strong enough inhibition to the DCN, and to code for different features of the movement. We believe that the firing rate is coding for movement features, and synchrony ensures the inhibition of the DCN.

\bigplot[analyses/figure3.png]{figure3}{
	\textbf{Short-term correlations of pairs of neighboring Purkinje cells during periods of increased firing during the task.}\\
	\textbf{A, E}: Rasters and firing rate time courses of the discharge around the time of ``cover'' for one session. Each line in the raster plots corresponds to one trial.
	\textbf{B}: Cross-correlogram of the cell pair shown in A during modulation (red line) and during the whole session (black line) normalized to match the average correlation during modulation in the $\pm20$ms window (bin=0.5ms). The time periods used to compute the cross-correlation are the union of the intervals of increased firing rate of both cells.
	\textbf{C}: Average cross-correlogram of all pairs of neighboring Purkinje cells during modulation. The standard error of the mean is shown in light red.
	\textbf{D}: Average cross-correlogram of pairs of distant cells during modulation, with SEM shown in light gray.
	\textbf{E}: Short-term correlations of neighboring cells are stronger during modulation than during the whole task. The correlations are plotted as a function of the absolute value of the delays between the spikes of the pairs of cells. Dots correspond to the average cross-correlation during modulation (bin 0.5ms); the red line is obtained form the points after smoothing with a Gaussian kernel with $\sigma=1$ms. The average cross-correlation during the whole task is shown in black. Each correlogram is normalized before pooling to the average correlation in a $\pm20$ms window. 
	\textbf{F, G}: Pairs of neighboring cells with a weak (\textbf{F}) or strong (\textbf{G}) co-modulation of the firing rates; correlations are plotted as a function of the absolute value of the delays between the spikes of the two cells.
}

\myparagraph{Significance of cross-correlations}
The traditional metric for assessing the significance of correlations in presence of changes in firing rate is to compare it against an expected level of non-specific correlation by shuffling the trials before computing the correlations. The drawback of this approach is the big amount of data that it requires, given its low statistical power. Since the analysis of synchrony requires spike trains to be as clean as possible, we had to remove the majority of recordings so that we could work on precisely sorted spike trains.  Due to this we deemed appropriate to use a more powerful statistical approach, that is the smoothing spline ANOVA model. The smoothing spline ANOVA models \cite{gu2013smoothing} are a family of non-parametric line fitting methods that use penalized log-likelihood minimization to explore statistical relations between multiple parameters and an independent variable. In our case, we used this method to obtain fits of single cells’ firing rates as a function of “internal” (time elapsed since last spike) and “external” (distance in time from landmark, delay to nearest spike of another cell) variables of the spike trains. To examine the firing coordination between cells, we first defined two models: the first one (complex model) predicted the probability of firing using all “internal” and “external” variables, while the second (simple model) didn't consider the ``delay to the nearest spike of another cell'' variable, effectively missing all information on other cells. We then evaluated the log-likelihood of the predictions of two models on a test set through 2-fold cross-validation, finding that the complex one performed better in the vast majority of neighbor couples where the models converged, producing better predictions than the simple model, and confirming the significance of the interactions between most neighboring cells. The implementation of the smoothing spline model that we used \cite{gu2014smoothing} allowed us to construct surrogate spike train from the produced fits, therefore we were able to inspect cross-correlations between a spike train from a real cell and a spike train produced by either the simple or the complex model (Figure \ref{figure4}F and G). 

\bigplot[analyses/figure4.png]{figure4}{
	\textbf{Modeling the firing rate of pairs of neighboring Purkinje cells with smoothing spines allows for the quantification of their temporal correlations.}\\
	Two models with smoothing splines were used: a simple model fitting the instantaneous firing probability as a function of the time in the trial $t$ and the time elapsed since the previous spike $t-t^{prev}$, and a complex model with an additional ``cell interaction delay'' term corresponding to the dilay to the nearest spike from a neighboring cell $t - t^{nearest}$, allowing the assessment of the temporal coordination between the cells.
	\textbf{A-E}: Logit-transformed probability of functions $\eta$ for the terms of the model, where the shading is the 95\% Bayesian confidence interval. Negative and positive values correspond respectively to a reduction and an increase of the instantaneous conditional firing probability. 
	\textbf{F-H}:Cross-correlations of the two real cells used to build the model (\textbf{F}) and of the first real cell and surrogate spike trains from the complex (\textbf{G}) and simple (\textbf{H}) model.
}

The cross-correlogram with the spike trains produced by the complex model showed a central peak of the same size as the one observed between real cells, while the spike train simulated from the simple model didn't exhibit cross-correlations. This provides a sanity-check on modeling, and demonstrate the ability to identify significant correlations at the level of single pairs of cells. With evidence of a significant interaction, and of the statistical and empirical validity of the model used, we proceeded to quantify the time scale of the interaction itself. To achieve this, we examined the term describing the predicted relation between probability of firing and delay from nearest spike of the neighboring cell, deriving it from the fitted complex models that achieved a better performance in the log-likelihood test. We then identified significantly positive peaks in the fits of this term, that indicate a significant increase of probability of firing produced by a neighboring spike at a given delay, and hence provide an estimate of the width of the temporal association of neighboring cell spikes.  In about 80\% of the neighboring couples, the temporal association with neighboring cells firing was significant for inter-spike delays shorter than 5 ms (Figure \ref{figure5}E), with a peak occurring a time less than 4 ms (Figure \ref{figure5}C).When applying the same pipeline on the small proportion of distant cells where the complex model was found to be a better fit, the distribution of the peaks of interaction delays were almost flat and not concentrated before 5 ms (Figure \ref{figure5}D). Overall, these results demonstrate that, during a reach-to-grasp movement, the large majority of Purkinje cells belonging to the same microzone fire in near synchrony with delays shorter than 5 ms.

\bigplot[analyses/figure5.png]{figure5}{
	\textbf{Significant near synchronous firing of most pairs of neighboring cells demonstrated with smoothing spline penalized likelihood models}\\
	\textbf{A}: In most cells, the firing probability is best predicted when taking into account the delay to the nearest spike of neighboring cells (complex model).
	\textbf{C}: Distribution of peak position for the cells' interaction term $\eta_{t-t^{\prime\ nearest}}$. Note that the time scale of positive interaction delays between the neighboring cells firing is clearly below 5ms.
	\textbf{E}: Distribution of ranges of significantly positive cells' interaction delay term. The results on the left column indicate that the firing probability of the cells is increased if a neighboring cell fires in a short window of a few milliseconds. This conclusion does not hold for distant cells (right column).
}
